\documentclass{article}

\usepackage{ctex}
\usepackage{amsmath}

\begin{document}
	\section{行内公式}
	\subsection{美元符号(常用)} 
		交换律是 $a+b=b+a$,如 $1+2=2+1=3$。
	\subsection{小括号}
		交换律是 \(a+b=b+a\),如 \(1+2=2+1=3\)。
	\subsection{math环境}
		交换律是 
		\begin{math}
			a+b=b+a
		\end{math},如
		\begin{math}
			1+2=2+1=3
		\end{math}。
	
	\section{上下标}
	\subsection{上标}
		$3x^{20} - x + 2 = 0$
	\subsection{下标}
		$a_0,a_1,...,a_{100}$
		
	\section{希腊字母}
		$\alpha$
		$\beta$
		$\gamma$
		$\epsilon$
		$\pi$
		$\omega$
		
		$\Gamma$
		$\Delta$
		$\Theta$
		$\Pi$
		$\Omega$
		
		$\alpha^2 + \beta^2 + \gamma = 0$
		
	\section{数学函数}
		$\sin$
		$\cos$
		$\log$
		$\ln$
		$\arcsin$
		$\arccos$
		
		$\sin^2 x + \cos^2 x = 1$
		
		$\sqrt[3]{x^2+y^2}$
		
	\section{分式}
		$3/4$
		$\frac{3}{4}$
		
	\section{行间公式}	
	\subsection{美元符号}
		交换律是 
		$$a+b=b+a$$
		如 
		$$1+2=2+1=3$$
	\subsection{中括号(常用)}
		交换律是 
		\[a+b=b+a\]
		如 
		\[1+2=2+1=3\]
	\subsection{displaymath环境}
		交换律是
		\begin{displaymath}
		a+b=b+a	
		\end{displaymath}
		如
		\begin{displaymath}
		1+2=2+1=3	
		\end{displaymath}
	\subsection{自动编号公式equation环境(常用)}
		交换律见式\ref{commutative}	%引用公式标签
		\begin{equation}
			a+b=b+a	\label{commutative}	%添加公式标签
		\end{equation}
	\subsection{不编号公式equation*环境}	%需要amsmath宏包
		交换律见式\ref{commutative2}	%引用公式标签
		\begin{equation*}
			a+b=b+a	\label{commutative2}	%添加公式标签
		\end{equation*}
	
\end{document}